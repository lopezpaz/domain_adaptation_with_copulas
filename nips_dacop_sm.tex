% 1. Relax Assumptions
% 2. Classification Problems
% 3. It is clear that dimension do not damage us?
% 4. Can we add Arcene?
% 5. Can we show nice things to avoid concern fro mcomputation time?
% 6. Scalability results.
% 7. Explanatory capabilities
% 8. How many factors are updated?
% 11. Performance with different number of samples
% 12. "Many details about the app to DA need to be filled"

\documentclass{article}

\usepackage{graphicx,
            amsmath,
            subfigure,
            amsfonts,
            tikz,
            bm}

\usepackage{nips12submit_e,times}

\usepackage[pdftex]{hyperref}

\renewcommand\arraystretch{1.2}
\newcommand{\f}[1]{\textbf{#1}}

\begin{document}

\title{Supplementary Material for \emph{Semi-Supervised Domain Adaptation with Non-Parametric Copulas}}
\nipsfinalcopy
\maketitle
\section*{Setup of Datasets for Domain Adaptation Experiments (Table 2)}
\begin{enumerate}
  \item \textbf{Wine:} 
    \begin{itemize}
      \item \textbf{URL:} \href{http://archive.ics.uci.edu/ml/datasets/Wine+Quality}{\texttt{http://archive.ics.uci.edu/ml/datasets/Wine+Quality}}
      \item \textbf{Description:} Twelve chemical properties of different wine brands are given as input vectors for each sample. As
            the ouput variable, an expert wine-taster quality score, ranging from 1 to 10.
      \item \textbf{Number of attributes:} 12.
      \item \textbf{Source task:} Predict white wines quality score.
      \item \textbf{Target task:} Predict red wines quality score.
    \end{itemize}
  \item \textbf{Sarcos:} 
    \begin{itemize}
      \item \textbf{URL:} \href{http://www.gaussianprocess.org/gpml/data/}{\texttt{http://www.gaussianprocess.org/gpml/data/}}
      \item \textbf{Description:} The data relates to an inverse dynamics problem for a seven degrees-of-freedom SARCOS anthropomorphic robot arm. The task is to map from a 21-dimensional input space (7 joint positions, 7 joint velocities, 7 joint accelerations) to the corresponding 7 joint torques. 
      \item \textbf{Number of attributes:} 21.
      \item \textbf{Source task:} Predict first torque.
      \item \textbf{Target task:} Predict last  torque.
    \end{itemize}
  \item \textbf{Rocks-Mines:} 
    \begin{itemize}
      \item \textbf{URL:} \href{http://archive.ics.uci.edu/ml/datasets/Connectionist+Bench+(Sonar,+Mines+vs.+Rocks)}{\texttt{http://archive.ics.uci.edu/ml/datasets/Connectionist...}}
      \item \textbf{Description:} Each pattern is a set of 60 numbers in the range 0.0 to 1.0. Each number represents the energy within a particular frequency band when bouncing sonar signals to either mines or rocks.
      \item \textbf{Number of attributes:} 60.
      \item \textbf{Source task:} Predict energy within the last frequency band read from mines.
      \item \textbf{Target task:} Predict energy within the last frequency band read from rocks.
    \end{itemize}
  \item \textbf{Hill-Valleys:} 
    \begin{itemize}
      \item \textbf{URL:} \href{http://archive.ics.uci.edu/ml/datasets/Hill-Valley}{\texttt{http://archive.ics.uci.edu/ml/datasets/Hill-Valley}}
      \item \textbf{Description:} Each record represents 100 points on a two-dimensional graph. When plotted in order (from 1 through 100) as the Y coordinate, the points will create either a Hill (a bump in the terrain) or a Valley (a dip in the terrain). 
      \item \textbf{Number of attributes:} 100.
      \item \textbf{Source task:} Predict altitude at the middle of a valley.
      \item \textbf{Target task:} Predict altitude at the middle of a hill.
    \end{itemize}
  \item \textbf{Axis-Slice:} 
    \begin{itemize}
      \item \textbf{URL:} \href{http://archive.ics.uci.edu/ml/datasets/Relative+location+of+CT+slices+on+axial+axis}{\texttt{http://archive.ics.uci.edu/ml/datasets/Relative...}}
      \item \textbf{Description:} The data was retrieved from a set of 53500 CT images from 74 different 
      patients (43 male, 31 female). Each CT slice is described by two histograms in polar space. 
      The first histogram describes the location of bone structures in the image, the second the location of air inclusions inside of the body.
      \item \textbf{Number of Attributes:} 386
      \item \textbf{Source task:} Predict first bone structure of female patients.
      \item \textbf{Target task:} Predict first bone structure of male patients.
    \end{itemize}
  \item \textbf{Isolet:} 
    \begin{itemize}
      \item \textbf{URL:} \href{http://archive.ics.uci.edu/ml/datasets/ISOLET}{\texttt{http://archive.ics.uci.edu/ml/datasets/ISOLET}}
      \item \textbf{Description:} 150 subjects spoke the name of each letter of the alphabet twice. The features include spectral coefficients; contour features, sonorant features, pre-sonorant features, and post-sonorant features. 
      \item \textbf{Number of Attributes:} 617
      \item \textbf{Source task:} Predict last post-sonorant feature from first half of speakers. 
      \item \textbf{Target task:} Predict last post-sonorant feature from second half of speakers.
    \end{itemize}
\end{enumerate}
\end{document}
